\section[Sub1]{Sub1}\label{main1:sub:sub1}

\nameref{main1:sub:sub2} shows a item list with different referencing. \autoref{lst:main1:sub:sub1} shows an example call to start the TeX reference checker.


\begin{figure}[!ht]
    \caption[{Main1; Sub; Sub1}]{Example Figure Sub1 \%}\label{fig:main1:sub:sub1}  % use to check \% masking comment symbol in texrefcheck
\end{figure}


The Listing below uses different tex conform label styles as list:
\begin{itemize}
    \item \ref{lst:main1:sub:sub1}
    \item \ref{lst:main1:sub:sub1:t2}
    \item \ref{lst:main1:sub:sub1:t3}
\end{itemize}

Here is more then one label checked in one line: \ref{lst:main1:sub:sub1}, \ref{lst:main1:sub:sub1:t4}.


\begin{lstlisting}[language=bash,caption={Start texrefchk (Label Typ 1)}, label={lst:main1:sub:sub1}]
./texrefchk.sh --texdir=./test/01_pass
\end{lstlisting}


\begin{lstlisting}[language=bash, label = lst:main1:sub:sub1:t2, caption={Start texrefchk (Label Typ 2)}, ]
./texrefchk.sh --texdir=./test/01_pass
\end{lstlisting}


\begin{lstlisting}[language=bash, label=lst:main1:sub:sub1:t3, caption={Start texrefchk (Label Typ 3)}]
./texrefchk.sh --texdir=./test/01_pass
\end{lstlisting}


\begin{lstlisting}[language=bash, label= lst:main1:sub:sub1:t4, caption={Start texrefchk (Label Typ 4)}]
./texrefchk.sh --texdir=./test/01_pass
\end{lstlisting}



% comment section below tries to check if comments in texrefcheck are honored
%
%\begin{figure}[!h]
%    \caption[{Main1; Sub; Sub1}]{Example Figure Sub1 \%}\label{fig:main1:sub:sub1}
%\end{figure}
%
% This setence references to a out-commented broken reference \autoref{outcomment:broken:ref}
%
